% 2016-01-27, author Luke L. Davis: This is the preamble for my shorthand commands,
% and various math constructions. Can be copied to beginning of document or use
% % 2016-01-27, author Luke L. Davis: This is the preamble for my shorthand commands,
% and various math constructions. Can be copied to beginning of document or use
% % 2016-01-27, author Luke L. Davis: This is the preamble for my shorthand commands,
% and various math constructions. Can be copied to beginning of document or use
% % 2016-01-27, author Luke L. Davis: This is the preamble for my shorthand commands,
% and various math constructions. Can be copied to beginning of document or use
% \input{\string~/preamble.tex} in header (should store in home directory)
%
% \newcommand notes: CANNOT create a newcommand with same name as an environment;
% will get weird behavior; instead use some shortcut/abbrev

% Bib notes: 
% 0) use \bibliography INSIDE BODY to PRINT bibliography in natbib (I think need to also declare in preamble... or not? actually, no!), or use \printbibliography in biblatex
% 1) can also use NONE, and \cite{}, \bibliography{} commands work when you run BIBTEX COMMAND on the file... right??? and, so don't have to run bibtex on file otherwise?
% 2a) for smaller bibliographies, use \frame[shrink=50]{\printbibliography or \bibliography{file}} if you are using bibtex (is allowable, but remember, natbib [which uses bibtex as backend] does not work in beamer)
% 2b) smaller bibliographies in article, use {\footnotesize OR \small \bibliography} when using natbib 	
% 3) for some biblatex styles, need backend=biber instead of bibtex (biber is much newer/better than bibtex)... but these need different formatted .bib files
% 4) in Zotero, use Cmd+Shift+C to copy BibTeX citations onto clipboard -- seriously awesome!

%------------------------------------------------------------------------------
% MOST IMPORTANT/FOR PAPERS
%------------------------------------------------------------------------------
%% Must be changed OFTEN for different classes (e.g. publisher-supplied classes/templates
%% or Bibliography interpreters [AGU doesnt require natbib or biblatex], slides vs. article)
\usepackage[utf8]{inputenc}  % disallows weird characters for unexpected behavior
\usepackage[T1]{fontenc}  % ?
\usepackage{url}  % package for formatting urls.
%\usepackage[backend=bibtex,style=alphabetic,citestyle=authoryear]{biblatex}  % more "recent", and works for beamer (natbib  NOT work with slides)
%\usepackage{natbib}  % allows citet, etc. and can use with custom .bst files; use for publications, and declare style with \bibliographystyle{<style>}
%\usepackage{biblatex}  % basic, for beamer, etc.
\usepackage{lmodern}  % this and ^^ declare font for whole article.
%\usepackage[margin=1in]{geometry}  % geometry allows easy margin-formatting
\usepackage[labelfont=bf]{caption}  % for better/more legible caption labels; very simple
\usepackage{subcaption}  % for things like subfigure, etc.
%\usepackage{float}  % special H location specifier is strong
\usepackage{graphicx}  % embedded bitmap, vector graphics
\usepackage{epstopdf}  % converts EPS to PDF files
\usepackage{url}  % package for formatting urls.
%------------------------------------------------------------------------------
% PACKAGES
%------------------------------------------------------------------------------
%% Important Packages
% For math, etc.
\usepackage{amsmath}
\usepackage{mathtools}  % fixes some amsmath quirks
\usepackage{amsthm}
\usepackage{amsopn}
\usepackage{amssymb}
\usepackage{amsfonts}
\usepackage{booktabs}  % for lines between equations, use \midrule within; also
  % required for Pandas DataFrame.to_latex() method
%% Optional packages, for smaller/less relavant things
\usepackage{multirow}  % for arrays; look this up
% \usepackage{enumerate}  % enumitem options, and \begin{enumerate}[I.] e.g.
\usepackage{enumitem}  % more complex enumeration (e.g. step 1, step 2, ...)
\usepackage{cancel}  % special "cancel" operator
\usepackage{nth}  % nth{1} is 1^st, etc.
\usepackage{pdflscape}  % lets you switch particular pages to landscape with \begin{landscape} / \end{landscape}. also sets pdf's "rotate" attribute to rotate the landscape'd page{\tiny }
%% Some more specialized packages; need to look these up again
%\usepackage{blkarray}
%\usepackage{relsize}
%\usepackage{pgffor}  % iteration of procedures
%\usepackage{setspace}  % enables setstretch amont other things
%\usepackage{xhfill}% http://ctan.org/pkg/xhfill
%\usepackage{titling}  % title options
%\usepackage[retainorgcmds]{IEEEtrantools}
%\usepackage{doc}  % special logo commands? or... glossary? what the heck is this one?

%------------------------------------------------------------------------------
% REMAINING SHORTHANDS THAT MAY BE USEFUL
%------------------------------------------------------------------------------
%% Simple math operators (determines proper formatting AND spacing)
% \DeclareMathOperator{\Lagr}{\mathcal{L}}  % cursive L
\DeclareMathOperator{\Tr}{trace}
\DeclareMathOperator{\id}{id}  % identity function? not sure
\DeclareMathOperator{\eff}{eff}  % effective? not sure
\DeclareMathOperator{\range}{range}
\DeclareMathOperator{\nulls}{null}  % nullspace
\DeclareMathOperator{\rank}{rank} 
\DeclareMathOperator{\sign}{sign}  % sign function
\DeclareMathOperator{\Span}{span}  % span
\DeclareMathOperator{\im}{Im}
% \DeclareMathOperator{\d}{d}
\DeclareMathOperator{\Char}{char}
\DeclareMathOperator{\argmax}{argmax}
\DeclareMathOperator{\argmin}{argmin}
\DeclareMathOperator{\sinc}{sinc}
%% Equation formatting
\newcommand{\numberthis}{\addtocounter{equation}{1}\tag{\theequation}}  % add to end of line
%% Options for assignments header standard format. capitalized start new lines and "Solution" prefix
% \newcommand{\sectionline}{\begin{center}\noindent\rule{\paperwidth/2}{0.4pt}\end{center}}  % better section line
\newcommand{\prob}[2]{\subsection*{Question #1}\textit{#2}\noindent}
\newcommand{\subprob}[2]{\subsubsection*{Part #1)}\textit{#2}\noindent}
\newcommand{\Prob}[2]{\subsection*{Question #1}\textit{#2}\\\\\textbf{Solution}: }
\newcommand{\Subprob}[2]{\subsubsection*{Part #1)}\textit{#2}\\\\\textbf{Solution}: }
%% Constants/units
\newcommand{\R}{8.314kg\,m^2\,s^{-2}\,mol^{-1}\,K^{-1}}
\newcommand{\kb}{1.38{\times}10^{-23}m^2\,kg\,s^{-2}\,K^{-1}}
\newcommand{\NA}{6.022{\times}10^{23}molec\,mol^{-1}}
\newcommand{\Rd}{287m^2\,s^{-2}\,K^{-1}}
\newcommand{\cpd}{1004J\,kg^{-1}\,K^{-1}}
\newcommand{\cvd}{717J\,kg^{-1}\,K^{-1}}
\newcommand{\cpw}{4187J\,kg^{-1}\,K^{-1}}
\newcommand{\cpi}{2108J\,kg^{-1}\,K^{-1}}
\newcommand{\atm}{1.01325 {\times} 10^5kg\,m^{-1}\,s^{-2}}
\newcommand{\speryr}{3.15576\E{7}s\,yr^{-1}}
\newcommand{\Pa}{kg\,m^{-1}\,s^{-2}}
\newcommand{\J}{kg\,m^2\,s^{-2}}
\newcommand{\rate}{cm^3\,molec^{-1}\,s^{-1}}  % rate constant
\newcommand{\nd}{molec\,cm^{-3}}  % number density
\newcommand{\pH}{\text{pH}}  % pH, without italics
%% Figures
\newcommand{\fig}[3]{%
  \begin{figure}[H]
  \centering 
  \includegraphics[width=#1]{#2}
  \caption{#3}
  \end{figure}%
  }  % papers, and not float package conflicts with AGUTeX
\newcommand{\Fig}[3]{%
  \begin{figure}[H]
  \centering 
  \makebox[\textwidth][c]{\includegraphics[width=#1]{#2}}
  \caption{#3}
  \end{figure}%
  }  % enforce centering always, even if runs into margins
\newcommand{\FIG}[3]{%
  \begin{minipage}{\linewidth}
  \centering 
  \makebox[\textwidth][c]{\includegraphics[width=#1]{#2}}
  \captionof{figure}{#3}  % can't just do figure
  \end{minipage}
  }  % enforce centering and disallows floating
\newcommand{\hFig}[3]{%
\begin{figure}
  \centering
  \begin{subfigure}{.5\textwidth}
    \centering
    \includegraphics[width=\linewidth]{#1}
  \end{subfigure}% necessary comment here, so figures are put side-by-side
  \begin{subfigure}{.5\textwidth}
    \centering
    \includegraphics[width=\linewidth]{#2}
  \end{subfigure}
  \caption{#3}
\end{figure}  % multiple Fig-style figures
}
\newcommand{\hFIG}[3]{%
  \begin{minipage}{\linewidth}
  \centering
  \begin{subfigure}{.5\textwidth}
    \centering
    \includegraphics[width=\linewidth]{#1}
  \end{subfigure}% necessary comment here, so figures are put side-by-side
  \begin{subfigure}{.5\textwidth}
    \centering
    \includegraphics[width=\linewidth]{#2}
  \end{subfigure}
    \captionof{figure}{#3}
\end{minipage}  % multiple FIG-style figures
}
\newcommand{\vFig}[3]{%
\begin{figure}
  \centering
  \begin{subfigure}{\textwidth}
    \centering
    \includegraphics[width=\linewidth]{#1}
  \end{subfigure}\\% put figures vertically
  \begin{subfigure}{\textwidth}
    \centering
    \includegraphics[width=\linewidth]{#2}
  \end{subfigure}
  \caption{#3}
\end{figure}  % multiple Fig-style figures
}
\newcommand{\vFIG}[3]{%
  \begin{minipage}{\linewidth}
  \centering
  \begin{subfigure}{\textwidth}
    \centering
    \includegraphics[width=\linewidth]{#1}
  \end{subfigure}\\% put figures vertically
  \begin{subfigure}{\textwidth}
    \centering
    \includegraphics[width=\linewidth]{#2}
  \end{subfigure}
    \captionof{figure}{#3}
\end{minipage}  % multiple FIG-style figures
}

%------------------------------------------------------------------------------
% EVERYTHING ELSE PROBABLY USELESS
%------------------------------------------------------------------------------
%------------------------------------------------------------------------------
% OPERATORS/SHORTHANDS
%------------------------------------------------------------------------------
%% Derivatives
\newcommand{\DD}[1][]{\,\text{D}#1}
\newcommand{\dd}[1][]{\,\text{d}#1}  % for dx, dy, etc.; it is an "operator" now
\newcommand{\Deriv}[3][]{\frac{\text{D}^{#1}#3}{\text{D} {#2}^{#1}}}  %capital D
\newcommand{\pderiv}[3][]{\frac{\partial^{#1}#3}{\partial {#2}^{#1}}}
\newcommand{\deriv}[3][]{\frac{\text{d}^{#1}#3}{\text{d} {#2}^{#1}}}
\newcommand{\px}{\frac{\partial}{\partial x}}
\newcommand{\py}{\frac{\partial}{\partial y}}
\newcommand{\pz}{\frac{\partial}{\partial z}}
\newcommand{\pt}{\frac{\partial}{\partial t}}
\newcommand{\dx}{\frac{\text{d}}{\text{d} x}}
\newcommand{\dy}{\frac{\text{d}}{\text{d} y}}
\newcommand{\dz}{\frac{\text{d}}{\text{d} z}}
\newcommand{\dt}{\frac{\text{d}}{\text{d} t}}
\newcommand{\Dx}{\frac{\text{D}}{\text{D} x}}
\newcommand{\Dy}{\frac{\text{D}}{\text{D} y}}
\newcommand{\Dz}{\frac{\text{D}}{\text{D} z}}
\newcommand{\Dt}{\frac{\text{D}}{\text{D} t}}
% Complex operators
\newcommand{\tnorm}[1]{\left\|#1\right\|_2}  % 2-norm (Euclidian norm)
\newcommand{\onorm}[1]{\left\|#1\right\|_1}  % 1-norm (sum of absolute values)
\newcommand{\inorm}[1]{\left\|#1\right\|_\infty}  % infinity norm
\newcommand{\limit}[2][\infty]{\lim_{#2\to#1} }  % limit, with default (symbol) goes to Inf
% Misc shorthands
\newcommand{\E}[1]{{\times}10^{#1}}
\newcommand{\Sum}{\sum_{i=1}^n}
\newcommand{\overbar}[1]{\overline{#1}}
\renewcommand{\tilde}[1]{\widetilde{#1}}  % tilde is too narrow
\renewcommand{\inf}{\infty}  % don't even know what \inf is; just redefine it
\newcommand{\ditto}[1][.4pt]{\text{\xrfill{#1}~\textquotedbl~\xrfill{#1}}}
\newcommand{\ubar}{\bar{u}}  % mean x wind
\newcommand{\vbar}{\bar{v}}  % mean y wind
\newcommand{\xbar}{\bar{x}}
\newcommand{\ybar}{\bar{y}}
\newcommand{\zbar}{\bar{z}}
\newcommand{\ihat}{\hat{i}}
\newcommand{\jhat}{\hat{j}}
\newcommand{\khat}{\hat{k}}
\newcommand{\uhat}{\hat{u}}  % unit x wind
\newcommand{\vhat}{\hat{v}}  % unit y wind
\newcommand{\xhat}{\hat{x}}  % unit x
\newcommand{\yhat}{\hat{y}}  % unit y
\newcommand{\that}{\hat{t}}  % unit
\newcommand{\fhat}{\hat{f}}  % unit Coriolis
\newcommand{\nhat}{\hat{n}}  % normal unit vector
\newcommand{\F }{{}^\circ F}  % degrees Fahrenheit
\newcommand{\C }{{}^\circ C}  % degrees Celcius
\newcommand{\Tau}{\mathrm{T}}  % greek tau
\newcommand{\given}{~|~}
\newcommand{\comma}{,~}
\newcommand{\fa}{~\forall~}
\newcommand{\ex}{~\exists~}

%------------------------------------------------------------------------------
% Environments, styles, formatting
%------------------------------------------------------------------------------
% Make arrays more spread out
\renewcommand{\arraystretch}{1.75}  % stretch arrays
\newcommand{\mytitle}[1]{\setstretch{.65}\maketitle}
\renewcommand{\label}[1]{\label{#1}\tag{#1}}
\def\[#1\]{\begin{align*}#1\end{align*}}  % turns \[ \] from \begin/end equation, to align
\renewcommand{\cases}[1]{\left\{\begin{array}{ll}#1\end{array}\right.}
\newcommand{\case}[1]{\left\{\begin{array}{ll}#1\end{array}\right.}
\newcommand{\pfrac}[2]{\left(\frac{#1}{#2}\right)}  % fraction with parentheses
\renewcommand{\frac}[2]{\dfrac{#1}{#2}}  % will always want display-fraction
\newcommand{\mat}[1]{\begin{pmatrix}#1\end{pmatrix}}  % quick parentheses matrix
\newcommand{\Mat}[1]{\begin{bmatrix}#1\end{bmatrix}}  % quick bracket matrix
\newcommand{\bullets}[1]{\begin{itemize}#1\end{itemize}}
% \newcommand{\numbers}[1]{\begin{enumerate}[1)]#1\end{enumerate}}
\newcommand{\numbers}[1]{\begin{enumerate}#1\end{enumerate}}
\renewcommand{\table}[2]{\begin{tabular}[#1]#2\end{tabular}}
\newcommand{\abc}[1]{\begin{enumerate}[a)]#1\end{enumerate}}
\renewcommand{\roman}[1]{\begin{enumerate}[I.]#1\end{enumerate}}
\newcommand{\terms}[1]{\begin{description}#1\end{description}}
% Styling shortcuts (math-mode and regular-mode)
\newcommand{\un}[1]{\underline{\smash{#1}}}  % underline without making line wider
\renewcommand{\it}[1]{\textit{#1}}  % italics
\renewcommand{\bf}[1]{\textbf{#1}}  % bold
\newcommand{\tit}[1]{\textit{#1}}  % italics
\newcommand{\tbf}[1]{\textbf{#1}}  % bold
\newcommand{\mb}[1]{\mathbf{#1}}  % math bold
\newcommand{\mcal}[1]{\mathcal{#1}}  % math calligraphy
\renewcommand{\t}[1]{\text{#1}}  % for quick text in math-mode
\newcommand{\tx}[1]{\text{#1}}  % for quick text in math-mode
\newcommand{\red}{\color{red}}  % enclose text in red color

%------------------------------------------------------------------------------
% Special
%------------------------------------------------------------------------------
% Field labels
\newcommand{\II}{\mathbb{I}}  % Imaginary
\newcommand{\RR}{\mathbb{R}}  % for Real numbers
\newcommand{\ZZ}{\mathbb{Z}}  % for Integers
\newcommand{\NN}{\mathbb{N}}  % et. cetera
\newcommand{\QQ}{\mathbb{Q}}
\newcommand{\CC}{\mathbb{C}}
\newcommand{\FF}{\mathbb{F}}
% Chemical constituents
\newcommand{\tF}{\text{F}}
\newcommand{\tX}{\text{X}}  % species "X"
\newcommand{\tCl}{\text{Cl}}
\newcommand{\tO}{\text{O}}  % oxygens
\newcommand{\tC}{\text{C}}  % carbon
\newcommand{\tN}{\text{N}}  % nitrogen
\newcommand{\tH}{\text{H}}  % hydrogen
\newcommand{\tS}{\text{S}}  % sulfur
\newcommand{\tM}{\text{M}}  % "third body"
\newcommand{\tR}{\text{R}}  % "hydrocarbon chain"
% Atmosphere common abbreviations
\newcommand{\TOA}{\text{TOA}}
\newcommand{\LW}{\text{LW}}
\newcommand{\SW}{\text{SW}}
\newcommand{\delt}{\partial_t}  % common partial derivatives
\newcommand{\deltt}{\partial_{tt}}
\newcommand{\delx}{\partial_x}
\newcommand{\dely}{\partial_y}
\newcommand{\delz}{\partial_z}
\newcommand{\delxx}{\partial_{xx}}
\newcommand{\delyy}{\partial_{yy}}
\newcommand{\delzz}{\partial_{zz}}
\newcommand{\delxy}{\partial_{xy}}
\newcommand{\delyx}{\partial_{yx}}
\newcommand{\delxz}{\partial_{xz}}
\newcommand{\delyz}{\partial_{yz}}
\newcommand{\deltx}{\partial_{tx}}
\newcommand{\delty}{\partial_{ty}}
\newcommand{\deltz}{\partial_{tz}}
\newcommand{\iter}[2]{#1^{(#2)}}  % iteration (for algorithm notation)
 in header (should store in home directory)
%
% \newcommand notes: CANNOT create a newcommand with same name as an environment;
% will get weird behavior; instead use some shortcut/abbrev

% Bib notes: 
% 0) use \bibliography INSIDE BODY to PRINT bibliography in natbib (I think need to also declare in preamble... or not? actually, no!), or use \printbibliography in biblatex
% 1) can also use NONE, and \cite{}, \bibliography{} commands work when you run BIBTEX COMMAND on the file... right??? and, so don't have to run bibtex on file otherwise?
% 2a) for smaller bibliographies, use \frame[shrink=50]{\printbibliography or \bibliography{file}} if you are using bibtex (is allowable, but remember, natbib [which uses bibtex as backend] does not work in beamer)
% 2b) smaller bibliographies in article, use {\footnotesize OR \small \bibliography} when using natbib 	
% 3) for some biblatex styles, need backend=biber instead of bibtex (biber is much newer/better than bibtex)... but these need different formatted .bib files
% 4) in Zotero, use Cmd+Shift+C to copy BibTeX citations onto clipboard -- seriously awesome!

%------------------------------------------------------------------------------
% MOST IMPORTANT/FOR PAPERS
%------------------------------------------------------------------------------
%% Must be changed OFTEN for different classes (e.g. publisher-supplied classes/templates
%% or Bibliography interpreters [AGU doesnt require natbib or biblatex], slides vs. article)
\usepackage[utf8]{inputenc}  % disallows weird characters for unexpected behavior
\usepackage[T1]{fontenc}  % ?
\usepackage{url}  % package for formatting urls.
%\usepackage[backend=bibtex,style=alphabetic,citestyle=authoryear]{biblatex}  % more "recent", and works for beamer (natbib  NOT work with slides)
%\usepackage{natbib}  % allows citet, etc. and can use with custom .bst files; use for publications, and declare style with \bibliographystyle{<style>}
%\usepackage{biblatex}  % basic, for beamer, etc.
\usepackage{lmodern}  % this and ^^ declare font for whole article.
%\usepackage[margin=1in]{geometry}  % geometry allows easy margin-formatting
\usepackage[labelfont=bf]{caption}  % for better/more legible caption labels; very simple
\usepackage{subcaption}  % for things like subfigure, etc.
%\usepackage{float}  % special H location specifier is strong
\usepackage{graphicx}  % embedded bitmap, vector graphics
\usepackage{epstopdf}  % converts EPS to PDF files
\usepackage{url}  % package for formatting urls.
%------------------------------------------------------------------------------
% PACKAGES
%------------------------------------------------------------------------------
%% Important Packages
% For math, etc.
\usepackage{amsmath}
\usepackage{mathtools}  % fixes some amsmath quirks
\usepackage{amsthm}
\usepackage{amsopn}
\usepackage{amssymb}
\usepackage{amsfonts}
\usepackage{booktabs}  % for lines between equations, use \midrule within; also
  % required for Pandas DataFrame.to_latex() method
%% Optional packages, for smaller/less relavant things
\usepackage{multirow}  % for arrays; look this up
% \usepackage{enumerate}  % enumitem options, and \begin{enumerate}[I.] e.g.
\usepackage{enumitem}  % more complex enumeration (e.g. step 1, step 2, ...)
\usepackage{cancel}  % special "cancel" operator
\usepackage{nth}  % nth{1} is 1^st, etc.
\usepackage{pdflscape}  % lets you switch particular pages to landscape with \begin{landscape} / \end{landscape}. also sets pdf's "rotate" attribute to rotate the landscape'd page{\tiny }
%% Some more specialized packages; need to look these up again
%\usepackage{blkarray}
%\usepackage{relsize}
%\usepackage{pgffor}  % iteration of procedures
%\usepackage{setspace}  % enables setstretch amont other things
%\usepackage{xhfill}% http://ctan.org/pkg/xhfill
%\usepackage{titling}  % title options
%\usepackage[retainorgcmds]{IEEEtrantools}
%\usepackage{doc}  % special logo commands? or... glossary? what the heck is this one?

%------------------------------------------------------------------------------
% REMAINING SHORTHANDS THAT MAY BE USEFUL
%------------------------------------------------------------------------------
%% Simple math operators (determines proper formatting AND spacing)
% \DeclareMathOperator{\Lagr}{\mathcal{L}}  % cursive L
\DeclareMathOperator{\Tr}{trace}
\DeclareMathOperator{\id}{id}  % identity function? not sure
\DeclareMathOperator{\eff}{eff}  % effective? not sure
\DeclareMathOperator{\range}{range}
\DeclareMathOperator{\nulls}{null}  % nullspace
\DeclareMathOperator{\rank}{rank} 
\DeclareMathOperator{\sign}{sign}  % sign function
\DeclareMathOperator{\Span}{span}  % span
\DeclareMathOperator{\im}{Im}
% \DeclareMathOperator{\d}{d}
\DeclareMathOperator{\Char}{char}
\DeclareMathOperator{\argmax}{argmax}
\DeclareMathOperator{\argmin}{argmin}
\DeclareMathOperator{\sinc}{sinc}
%% Equation formatting
\newcommand{\numberthis}{\addtocounter{equation}{1}\tag{\theequation}}  % add to end of line
%% Options for assignments header standard format. capitalized start new lines and "Solution" prefix
% \newcommand{\sectionline}{\begin{center}\noindent\rule{\paperwidth/2}{0.4pt}\end{center}}  % better section line
\newcommand{\prob}[2]{\subsection*{Question #1}\textit{#2}\noindent}
\newcommand{\subprob}[2]{\subsubsection*{Part #1)}\textit{#2}\noindent}
\newcommand{\Prob}[2]{\subsection*{Question #1}\textit{#2}\\\\\textbf{Solution}: }
\newcommand{\Subprob}[2]{\subsubsection*{Part #1)}\textit{#2}\\\\\textbf{Solution}: }
%% Constants/units
\newcommand{\R}{8.314kg\,m^2\,s^{-2}\,mol^{-1}\,K^{-1}}
\newcommand{\kb}{1.38{\times}10^{-23}m^2\,kg\,s^{-2}\,K^{-1}}
\newcommand{\NA}{6.022{\times}10^{23}molec\,mol^{-1}}
\newcommand{\Rd}{287m^2\,s^{-2}\,K^{-1}}
\newcommand{\cpd}{1004J\,kg^{-1}\,K^{-1}}
\newcommand{\cvd}{717J\,kg^{-1}\,K^{-1}}
\newcommand{\cpw}{4187J\,kg^{-1}\,K^{-1}}
\newcommand{\cpi}{2108J\,kg^{-1}\,K^{-1}}
\newcommand{\atm}{1.01325 {\times} 10^5kg\,m^{-1}\,s^{-2}}
\newcommand{\speryr}{3.15576\E{7}s\,yr^{-1}}
\newcommand{\Pa}{kg\,m^{-1}\,s^{-2}}
\newcommand{\J}{kg\,m^2\,s^{-2}}
\newcommand{\rate}{cm^3\,molec^{-1}\,s^{-1}}  % rate constant
\newcommand{\nd}{molec\,cm^{-3}}  % number density
\newcommand{\pH}{\text{pH}}  % pH, without italics
%% Figures
\newcommand{\fig}[3]{%
  \begin{figure}[H]
  \centering 
  \includegraphics[width=#1]{#2}
  \caption{#3}
  \end{figure}%
  }  % papers, and not float package conflicts with AGUTeX
\newcommand{\Fig}[3]{%
  \begin{figure}[H]
  \centering 
  \makebox[\textwidth][c]{\includegraphics[width=#1]{#2}}
  \caption{#3}
  \end{figure}%
  }  % enforce centering always, even if runs into margins
\newcommand{\FIG}[3]{%
  \begin{minipage}{\linewidth}
  \centering 
  \makebox[\textwidth][c]{\includegraphics[width=#1]{#2}}
  \captionof{figure}{#3}  % can't just do figure
  \end{minipage}
  }  % enforce centering and disallows floating
\newcommand{\hFig}[3]{%
\begin{figure}
  \centering
  \begin{subfigure}{.5\textwidth}
    \centering
    \includegraphics[width=\linewidth]{#1}
  \end{subfigure}% necessary comment here, so figures are put side-by-side
  \begin{subfigure}{.5\textwidth}
    \centering
    \includegraphics[width=\linewidth]{#2}
  \end{subfigure}
  \caption{#3}
\end{figure}  % multiple Fig-style figures
}
\newcommand{\hFIG}[3]{%
  \begin{minipage}{\linewidth}
  \centering
  \begin{subfigure}{.5\textwidth}
    \centering
    \includegraphics[width=\linewidth]{#1}
  \end{subfigure}% necessary comment here, so figures are put side-by-side
  \begin{subfigure}{.5\textwidth}
    \centering
    \includegraphics[width=\linewidth]{#2}
  \end{subfigure}
    \captionof{figure}{#3}
\end{minipage}  % multiple FIG-style figures
}
\newcommand{\vFig}[3]{%
\begin{figure}
  \centering
  \begin{subfigure}{\textwidth}
    \centering
    \includegraphics[width=\linewidth]{#1}
  \end{subfigure}\\% put figures vertically
  \begin{subfigure}{\textwidth}
    \centering
    \includegraphics[width=\linewidth]{#2}
  \end{subfigure}
  \caption{#3}
\end{figure}  % multiple Fig-style figures
}
\newcommand{\vFIG}[3]{%
  \begin{minipage}{\linewidth}
  \centering
  \begin{subfigure}{\textwidth}
    \centering
    \includegraphics[width=\linewidth]{#1}
  \end{subfigure}\\% put figures vertically
  \begin{subfigure}{\textwidth}
    \centering
    \includegraphics[width=\linewidth]{#2}
  \end{subfigure}
    \captionof{figure}{#3}
\end{minipage}  % multiple FIG-style figures
}

%------------------------------------------------------------------------------
% EVERYTHING ELSE PROBABLY USELESS
%------------------------------------------------------------------------------
%------------------------------------------------------------------------------
% OPERATORS/SHORTHANDS
%------------------------------------------------------------------------------
%% Derivatives
\newcommand{\DD}[1][]{\,\text{D}#1}
\newcommand{\dd}[1][]{\,\text{d}#1}  % for dx, dy, etc.; it is an "operator" now
\newcommand{\Deriv}[3][]{\frac{\text{D}^{#1}#3}{\text{D} {#2}^{#1}}}  %capital D
\newcommand{\pderiv}[3][]{\frac{\partial^{#1}#3}{\partial {#2}^{#1}}}
\newcommand{\deriv}[3][]{\frac{\text{d}^{#1}#3}{\text{d} {#2}^{#1}}}
\newcommand{\px}{\frac{\partial}{\partial x}}
\newcommand{\py}{\frac{\partial}{\partial y}}
\newcommand{\pz}{\frac{\partial}{\partial z}}
\newcommand{\pt}{\frac{\partial}{\partial t}}
\newcommand{\dx}{\frac{\text{d}}{\text{d} x}}
\newcommand{\dy}{\frac{\text{d}}{\text{d} y}}
\newcommand{\dz}{\frac{\text{d}}{\text{d} z}}
\newcommand{\dt}{\frac{\text{d}}{\text{d} t}}
\newcommand{\Dx}{\frac{\text{D}}{\text{D} x}}
\newcommand{\Dy}{\frac{\text{D}}{\text{D} y}}
\newcommand{\Dz}{\frac{\text{D}}{\text{D} z}}
\newcommand{\Dt}{\frac{\text{D}}{\text{D} t}}
% Complex operators
\newcommand{\tnorm}[1]{\left\|#1\right\|_2}  % 2-norm (Euclidian norm)
\newcommand{\onorm}[1]{\left\|#1\right\|_1}  % 1-norm (sum of absolute values)
\newcommand{\inorm}[1]{\left\|#1\right\|_\infty}  % infinity norm
\newcommand{\limit}[2][\infty]{\lim_{#2\to#1} }  % limit, with default (symbol) goes to Inf
% Misc shorthands
\newcommand{\E}[1]{{\times}10^{#1}}
\newcommand{\Sum}{\sum_{i=1}^n}
\newcommand{\overbar}[1]{\overline{#1}}
\renewcommand{\tilde}[1]{\widetilde{#1}}  % tilde is too narrow
\renewcommand{\inf}{\infty}  % don't even know what \inf is; just redefine it
\newcommand{\ditto}[1][.4pt]{\text{\xrfill{#1}~\textquotedbl~\xrfill{#1}}}
\newcommand{\ubar}{\bar{u}}  % mean x wind
\newcommand{\vbar}{\bar{v}}  % mean y wind
\newcommand{\xbar}{\bar{x}}
\newcommand{\ybar}{\bar{y}}
\newcommand{\zbar}{\bar{z}}
\newcommand{\ihat}{\hat{i}}
\newcommand{\jhat}{\hat{j}}
\newcommand{\khat}{\hat{k}}
\newcommand{\uhat}{\hat{u}}  % unit x wind
\newcommand{\vhat}{\hat{v}}  % unit y wind
\newcommand{\xhat}{\hat{x}}  % unit x
\newcommand{\yhat}{\hat{y}}  % unit y
\newcommand{\that}{\hat{t}}  % unit
\newcommand{\fhat}{\hat{f}}  % unit Coriolis
\newcommand{\nhat}{\hat{n}}  % normal unit vector
\newcommand{\F }{{}^\circ F}  % degrees Fahrenheit
\newcommand{\C }{{}^\circ C}  % degrees Celcius
\newcommand{\Tau}{\mathrm{T}}  % greek tau
\newcommand{\given}{~|~}
\newcommand{\comma}{,~}
\newcommand{\fa}{~\forall~}
\newcommand{\ex}{~\exists~}

%------------------------------------------------------------------------------
% Environments, styles, formatting
%------------------------------------------------------------------------------
% Make arrays more spread out
\renewcommand{\arraystretch}{1.75}  % stretch arrays
\newcommand{\mytitle}[1]{\setstretch{.65}\maketitle}
\renewcommand{\label}[1]{\label{#1}\tag{#1}}
\def\[#1\]{\begin{align*}#1\end{align*}}  % turns \[ \] from \begin/end equation, to align
\renewcommand{\cases}[1]{\left\{\begin{array}{ll}#1\end{array}\right.}
\newcommand{\case}[1]{\left\{\begin{array}{ll}#1\end{array}\right.}
\newcommand{\pfrac}[2]{\left(\frac{#1}{#2}\right)}  % fraction with parentheses
\renewcommand{\frac}[2]{\dfrac{#1}{#2}}  % will always want display-fraction
\newcommand{\mat}[1]{\begin{pmatrix}#1\end{pmatrix}}  % quick parentheses matrix
\newcommand{\Mat}[1]{\begin{bmatrix}#1\end{bmatrix}}  % quick bracket matrix
\newcommand{\bullets}[1]{\begin{itemize}#1\end{itemize}}
% \newcommand{\numbers}[1]{\begin{enumerate}[1)]#1\end{enumerate}}
\newcommand{\numbers}[1]{\begin{enumerate}#1\end{enumerate}}
\renewcommand{\table}[2]{\begin{tabular}[#1]#2\end{tabular}}
\newcommand{\abc}[1]{\begin{enumerate}[a)]#1\end{enumerate}}
\renewcommand{\roman}[1]{\begin{enumerate}[I.]#1\end{enumerate}}
\newcommand{\terms}[1]{\begin{description}#1\end{description}}
% Styling shortcuts (math-mode and regular-mode)
\newcommand{\un}[1]{\underline{\smash{#1}}}  % underline without making line wider
\renewcommand{\it}[1]{\textit{#1}}  % italics
\renewcommand{\bf}[1]{\textbf{#1}}  % bold
\newcommand{\tit}[1]{\textit{#1}}  % italics
\newcommand{\tbf}[1]{\textbf{#1}}  % bold
\newcommand{\mb}[1]{\mathbf{#1}}  % math bold
\newcommand{\mcal}[1]{\mathcal{#1}}  % math calligraphy
\renewcommand{\t}[1]{\text{#1}}  % for quick text in math-mode
\newcommand{\tx}[1]{\text{#1}}  % for quick text in math-mode
\newcommand{\red}{\color{red}}  % enclose text in red color

%------------------------------------------------------------------------------
% Special
%------------------------------------------------------------------------------
% Field labels
\newcommand{\II}{\mathbb{I}}  % Imaginary
\newcommand{\RR}{\mathbb{R}}  % for Real numbers
\newcommand{\ZZ}{\mathbb{Z}}  % for Integers
\newcommand{\NN}{\mathbb{N}}  % et. cetera
\newcommand{\QQ}{\mathbb{Q}}
\newcommand{\CC}{\mathbb{C}}
\newcommand{\FF}{\mathbb{F}}
% Chemical constituents
\newcommand{\tF}{\text{F}}
\newcommand{\tX}{\text{X}}  % species "X"
\newcommand{\tCl}{\text{Cl}}
\newcommand{\tO}{\text{O}}  % oxygens
\newcommand{\tC}{\text{C}}  % carbon
\newcommand{\tN}{\text{N}}  % nitrogen
\newcommand{\tH}{\text{H}}  % hydrogen
\newcommand{\tS}{\text{S}}  % sulfur
\newcommand{\tM}{\text{M}}  % "third body"
\newcommand{\tR}{\text{R}}  % "hydrocarbon chain"
% Atmosphere common abbreviations
\newcommand{\TOA}{\text{TOA}}
\newcommand{\LW}{\text{LW}}
\newcommand{\SW}{\text{SW}}
\newcommand{\delt}{\partial_t}  % common partial derivatives
\newcommand{\deltt}{\partial_{tt}}
\newcommand{\delx}{\partial_x}
\newcommand{\dely}{\partial_y}
\newcommand{\delz}{\partial_z}
\newcommand{\delxx}{\partial_{xx}}
\newcommand{\delyy}{\partial_{yy}}
\newcommand{\delzz}{\partial_{zz}}
\newcommand{\delxy}{\partial_{xy}}
\newcommand{\delyx}{\partial_{yx}}
\newcommand{\delxz}{\partial_{xz}}
\newcommand{\delyz}{\partial_{yz}}
\newcommand{\deltx}{\partial_{tx}}
\newcommand{\delty}{\partial_{ty}}
\newcommand{\deltz}{\partial_{tz}}
\newcommand{\iter}[2]{#1^{(#2)}}  % iteration (for algorithm notation)
 in header (should store in home directory)
%
% \newcommand notes: CANNOT create a newcommand with same name as an environment;
% will get weird behavior; instead use some shortcut/abbrev

% Bib notes: 
% 0) use \bibliography INSIDE BODY to PRINT bibliography in natbib (I think need to also declare in preamble... or not? actually, no!), or use \printbibliography in biblatex
% 1) can also use NONE, and \cite{}, \bibliography{} commands work when you run BIBTEX COMMAND on the file... right??? and, so don't have to run bibtex on file otherwise?
% 2a) for smaller bibliographies, use \frame[shrink=50]{\printbibliography or \bibliography{file}} if you are using bibtex (is allowable, but remember, natbib [which uses bibtex as backend] does not work in beamer)
% 2b) smaller bibliographies in article, use {\footnotesize OR \small \bibliography} when using natbib 	
% 3) for some biblatex styles, need backend=biber instead of bibtex (biber is much newer/better than bibtex)... but these need different formatted .bib files
% 4) in Zotero, use Cmd+Shift+C to copy BibTeX citations onto clipboard -- seriously awesome!

%------------------------------------------------------------------------------
% MOST IMPORTANT/FOR PAPERS
%------------------------------------------------------------------------------
%% Must be changed OFTEN for different classes (e.g. publisher-supplied classes/templates
%% or Bibliography interpreters [AGU doesnt require natbib or biblatex], slides vs. article)
\usepackage[utf8]{inputenc}  % disallows weird characters for unexpected behavior
\usepackage[T1]{fontenc}  % ?
\usepackage{url}  % package for formatting urls.
%\usepackage[backend=bibtex,style=alphabetic,citestyle=authoryear]{biblatex}  % more "recent", and works for beamer (natbib  NOT work with slides)
%\usepackage{natbib}  % allows citet, etc. and can use with custom .bst files; use for publications, and declare style with \bibliographystyle{<style>}
%\usepackage{biblatex}  % basic, for beamer, etc.
\usepackage{lmodern}  % this and ^^ declare font for whole article.
%\usepackage[margin=1in]{geometry}  % geometry allows easy margin-formatting
\usepackage[labelfont=bf]{caption}  % for better/more legible caption labels; very simple
\usepackage{subcaption}  % for things like subfigure, etc.
%\usepackage{float}  % special H location specifier is strong
\usepackage{graphicx}  % embedded bitmap, vector graphics
\usepackage{epstopdf}  % converts EPS to PDF files
\usepackage{url}  % package for formatting urls.
%------------------------------------------------------------------------------
% PACKAGES
%------------------------------------------------------------------------------
%% Important Packages
% For math, etc.
\usepackage{amsmath}
\usepackage{mathtools}  % fixes some amsmath quirks
\usepackage{amsthm}
\usepackage{amsopn}
\usepackage{amssymb}
\usepackage{amsfonts}
\usepackage{booktabs}  % for lines between equations, use \midrule within; also
  % required for Pandas DataFrame.to_latex() method
%% Optional packages, for smaller/less relavant things
\usepackage{multirow}  % for arrays; look this up
% \usepackage{enumerate}  % enumitem options, and \begin{enumerate}[I.] e.g.
\usepackage{enumitem}  % more complex enumeration (e.g. step 1, step 2, ...)
\usepackage{cancel}  % special "cancel" operator
\usepackage{nth}  % nth{1} is 1^st, etc.
\usepackage{pdflscape}  % lets you switch particular pages to landscape with \begin{landscape} / \end{landscape}. also sets pdf's "rotate" attribute to rotate the landscape'd page{\tiny }
%% Some more specialized packages; need to look these up again
%\usepackage{blkarray}
%\usepackage{relsize}
%\usepackage{pgffor}  % iteration of procedures
%\usepackage{setspace}  % enables setstretch amont other things
%\usepackage{xhfill}% http://ctan.org/pkg/xhfill
%\usepackage{titling}  % title options
%\usepackage[retainorgcmds]{IEEEtrantools}
%\usepackage{doc}  % special logo commands? or... glossary? what the heck is this one?

%------------------------------------------------------------------------------
% REMAINING SHORTHANDS THAT MAY BE USEFUL
%------------------------------------------------------------------------------
%% Simple math operators (determines proper formatting AND spacing)
% \DeclareMathOperator{\Lagr}{\mathcal{L}}  % cursive L
\DeclareMathOperator{\Tr}{trace}
\DeclareMathOperator{\id}{id}  % identity function? not sure
\DeclareMathOperator{\eff}{eff}  % effective? not sure
\DeclareMathOperator{\range}{range}
\DeclareMathOperator{\nulls}{null}  % nullspace
\DeclareMathOperator{\rank}{rank} 
\DeclareMathOperator{\sign}{sign}  % sign function
\DeclareMathOperator{\Span}{span}  % span
\DeclareMathOperator{\im}{Im}
% \DeclareMathOperator{\d}{d}
\DeclareMathOperator{\Char}{char}
\DeclareMathOperator{\argmax}{argmax}
\DeclareMathOperator{\argmin}{argmin}
\DeclareMathOperator{\sinc}{sinc}
%% Equation formatting
\newcommand{\numberthis}{\addtocounter{equation}{1}\tag{\theequation}}  % add to end of line
%% Options for assignments header standard format. capitalized start new lines and "Solution" prefix
% \newcommand{\sectionline}{\begin{center}\noindent\rule{\paperwidth/2}{0.4pt}\end{center}}  % better section line
\newcommand{\prob}[2]{\subsection*{Question #1}\textit{#2}\noindent}
\newcommand{\subprob}[2]{\subsubsection*{Part #1)}\textit{#2}\noindent}
\newcommand{\Prob}[2]{\subsection*{Question #1}\textit{#2}\\\\\textbf{Solution}: }
\newcommand{\Subprob}[2]{\subsubsection*{Part #1)}\textit{#2}\\\\\textbf{Solution}: }
%% Constants/units
\newcommand{\R}{8.314kg\,m^2\,s^{-2}\,mol^{-1}\,K^{-1}}
\newcommand{\kb}{1.38{\times}10^{-23}m^2\,kg\,s^{-2}\,K^{-1}}
\newcommand{\NA}{6.022{\times}10^{23}molec\,mol^{-1}}
\newcommand{\Rd}{287m^2\,s^{-2}\,K^{-1}}
\newcommand{\cpd}{1004J\,kg^{-1}\,K^{-1}}
\newcommand{\cvd}{717J\,kg^{-1}\,K^{-1}}
\newcommand{\cpw}{4187J\,kg^{-1}\,K^{-1}}
\newcommand{\cpi}{2108J\,kg^{-1}\,K^{-1}}
\newcommand{\atm}{1.01325 {\times} 10^5kg\,m^{-1}\,s^{-2}}
\newcommand{\speryr}{3.15576\E{7}s\,yr^{-1}}
\newcommand{\Pa}{kg\,m^{-1}\,s^{-2}}
\newcommand{\J}{kg\,m^2\,s^{-2}}
\newcommand{\rate}{cm^3\,molec^{-1}\,s^{-1}}  % rate constant
\newcommand{\nd}{molec\,cm^{-3}}  % number density
\newcommand{\pH}{\text{pH}}  % pH, without italics
%% Figures
\newcommand{\fig}[3]{%
  \begin{figure}[H]
  \centering 
  \includegraphics[width=#1]{#2}
  \caption{#3}
  \end{figure}%
  }  % papers, and not float package conflicts with AGUTeX
\newcommand{\Fig}[3]{%
  \begin{figure}[H]
  \centering 
  \makebox[\textwidth][c]{\includegraphics[width=#1]{#2}}
  \caption{#3}
  \end{figure}%
  }  % enforce centering always, even if runs into margins
\newcommand{\FIG}[3]{%
  \begin{minipage}{\linewidth}
  \centering 
  \makebox[\textwidth][c]{\includegraphics[width=#1]{#2}}
  \captionof{figure}{#3}  % can't just do figure
  \end{minipage}
  }  % enforce centering and disallows floating
\newcommand{\hFig}[3]{%
\begin{figure}
  \centering
  \begin{subfigure}{.5\textwidth}
    \centering
    \includegraphics[width=\linewidth]{#1}
  \end{subfigure}% necessary comment here, so figures are put side-by-side
  \begin{subfigure}{.5\textwidth}
    \centering
    \includegraphics[width=\linewidth]{#2}
  \end{subfigure}
  \caption{#3}
\end{figure}  % multiple Fig-style figures
}
\newcommand{\hFIG}[3]{%
  \begin{minipage}{\linewidth}
  \centering
  \begin{subfigure}{.5\textwidth}
    \centering
    \includegraphics[width=\linewidth]{#1}
  \end{subfigure}% necessary comment here, so figures are put side-by-side
  \begin{subfigure}{.5\textwidth}
    \centering
    \includegraphics[width=\linewidth]{#2}
  \end{subfigure}
    \captionof{figure}{#3}
\end{minipage}  % multiple FIG-style figures
}
\newcommand{\vFig}[3]{%
\begin{figure}
  \centering
  \begin{subfigure}{\textwidth}
    \centering
    \includegraphics[width=\linewidth]{#1}
  \end{subfigure}\\% put figures vertically
  \begin{subfigure}{\textwidth}
    \centering
    \includegraphics[width=\linewidth]{#2}
  \end{subfigure}
  \caption{#3}
\end{figure}  % multiple Fig-style figures
}
\newcommand{\vFIG}[3]{%
  \begin{minipage}{\linewidth}
  \centering
  \begin{subfigure}{\textwidth}
    \centering
    \includegraphics[width=\linewidth]{#1}
  \end{subfigure}\\% put figures vertically
  \begin{subfigure}{\textwidth}
    \centering
    \includegraphics[width=\linewidth]{#2}
  \end{subfigure}
    \captionof{figure}{#3}
\end{minipage}  % multiple FIG-style figures
}

%------------------------------------------------------------------------------
% EVERYTHING ELSE PROBABLY USELESS
%------------------------------------------------------------------------------
%------------------------------------------------------------------------------
% OPERATORS/SHORTHANDS
%------------------------------------------------------------------------------
%% Derivatives
\newcommand{\DD}[1][]{\,\text{D}#1}
\newcommand{\dd}[1][]{\,\text{d}#1}  % for dx, dy, etc.; it is an "operator" now
\newcommand{\Deriv}[3][]{\frac{\text{D}^{#1}#3}{\text{D} {#2}^{#1}}}  %capital D
\newcommand{\pderiv}[3][]{\frac{\partial^{#1}#3}{\partial {#2}^{#1}}}
\newcommand{\deriv}[3][]{\frac{\text{d}^{#1}#3}{\text{d} {#2}^{#1}}}
\newcommand{\px}{\frac{\partial}{\partial x}}
\newcommand{\py}{\frac{\partial}{\partial y}}
\newcommand{\pz}{\frac{\partial}{\partial z}}
\newcommand{\pt}{\frac{\partial}{\partial t}}
\newcommand{\dx}{\frac{\text{d}}{\text{d} x}}
\newcommand{\dy}{\frac{\text{d}}{\text{d} y}}
\newcommand{\dz}{\frac{\text{d}}{\text{d} z}}
\newcommand{\dt}{\frac{\text{d}}{\text{d} t}}
\newcommand{\Dx}{\frac{\text{D}}{\text{D} x}}
\newcommand{\Dy}{\frac{\text{D}}{\text{D} y}}
\newcommand{\Dz}{\frac{\text{D}}{\text{D} z}}
\newcommand{\Dt}{\frac{\text{D}}{\text{D} t}}
% Complex operators
\newcommand{\tnorm}[1]{\left\|#1\right\|_2}  % 2-norm (Euclidian norm)
\newcommand{\onorm}[1]{\left\|#1\right\|_1}  % 1-norm (sum of absolute values)
\newcommand{\inorm}[1]{\left\|#1\right\|_\infty}  % infinity norm
\newcommand{\limit}[2][\infty]{\lim_{#2\to#1} }  % limit, with default (symbol) goes to Inf
% Misc shorthands
\newcommand{\E}[1]{{\times}10^{#1}}
\newcommand{\Sum}{\sum_{i=1}^n}
\newcommand{\overbar}[1]{\overline{#1}}
\renewcommand{\tilde}[1]{\widetilde{#1}}  % tilde is too narrow
\renewcommand{\inf}{\infty}  % don't even know what \inf is; just redefine it
\newcommand{\ditto}[1][.4pt]{\text{\xrfill{#1}~\textquotedbl~\xrfill{#1}}}
\newcommand{\ubar}{\bar{u}}  % mean x wind
\newcommand{\vbar}{\bar{v}}  % mean y wind
\newcommand{\xbar}{\bar{x}}
\newcommand{\ybar}{\bar{y}}
\newcommand{\zbar}{\bar{z}}
\newcommand{\ihat}{\hat{i}}
\newcommand{\jhat}{\hat{j}}
\newcommand{\khat}{\hat{k}}
\newcommand{\uhat}{\hat{u}}  % unit x wind
\newcommand{\vhat}{\hat{v}}  % unit y wind
\newcommand{\xhat}{\hat{x}}  % unit x
\newcommand{\yhat}{\hat{y}}  % unit y
\newcommand{\that}{\hat{t}}  % unit
\newcommand{\fhat}{\hat{f}}  % unit Coriolis
\newcommand{\nhat}{\hat{n}}  % normal unit vector
\newcommand{\F }{{}^\circ F}  % degrees Fahrenheit
\newcommand{\C }{{}^\circ C}  % degrees Celcius
\newcommand{\Tau}{\mathrm{T}}  % greek tau
\newcommand{\given}{~|~}
\newcommand{\comma}{,~}
\newcommand{\fa}{~\forall~}
\newcommand{\ex}{~\exists~}

%------------------------------------------------------------------------------
% Environments, styles, formatting
%------------------------------------------------------------------------------
% Make arrays more spread out
\renewcommand{\arraystretch}{1.75}  % stretch arrays
\newcommand{\mytitle}[1]{\setstretch{.65}\maketitle}
\renewcommand{\label}[1]{\label{#1}\tag{#1}}
\def\[#1\]{\begin{align*}#1\end{align*}}  % turns \[ \] from \begin/end equation, to align
\renewcommand{\cases}[1]{\left\{\begin{array}{ll}#1\end{array}\right.}
\newcommand{\case}[1]{\left\{\begin{array}{ll}#1\end{array}\right.}
\newcommand{\pfrac}[2]{\left(\frac{#1}{#2}\right)}  % fraction with parentheses
\renewcommand{\frac}[2]{\dfrac{#1}{#2}}  % will always want display-fraction
\newcommand{\mat}[1]{\begin{pmatrix}#1\end{pmatrix}}  % quick parentheses matrix
\newcommand{\Mat}[1]{\begin{bmatrix}#1\end{bmatrix}}  % quick bracket matrix
\newcommand{\bullets}[1]{\begin{itemize}#1\end{itemize}}
% \newcommand{\numbers}[1]{\begin{enumerate}[1)]#1\end{enumerate}}
\newcommand{\numbers}[1]{\begin{enumerate}#1\end{enumerate}}
\renewcommand{\table}[2]{\begin{tabular}[#1]#2\end{tabular}}
\newcommand{\abc}[1]{\begin{enumerate}[a)]#1\end{enumerate}}
\renewcommand{\roman}[1]{\begin{enumerate}[I.]#1\end{enumerate}}
\newcommand{\terms}[1]{\begin{description}#1\end{description}}
% Styling shortcuts (math-mode and regular-mode)
\newcommand{\un}[1]{\underline{\smash{#1}}}  % underline without making line wider
\renewcommand{\it}[1]{\textit{#1}}  % italics
\renewcommand{\bf}[1]{\textbf{#1}}  % bold
\newcommand{\tit}[1]{\textit{#1}}  % italics
\newcommand{\tbf}[1]{\textbf{#1}}  % bold
\newcommand{\mb}[1]{\mathbf{#1}}  % math bold
\newcommand{\mcal}[1]{\mathcal{#1}}  % math calligraphy
\renewcommand{\t}[1]{\text{#1}}  % for quick text in math-mode
\newcommand{\tx}[1]{\text{#1}}  % for quick text in math-mode
\newcommand{\red}{\color{red}}  % enclose text in red color

%------------------------------------------------------------------------------
% Special
%------------------------------------------------------------------------------
% Field labels
\newcommand{\II}{\mathbb{I}}  % Imaginary
\newcommand{\RR}{\mathbb{R}}  % for Real numbers
\newcommand{\ZZ}{\mathbb{Z}}  % for Integers
\newcommand{\NN}{\mathbb{N}}  % et. cetera
\newcommand{\QQ}{\mathbb{Q}}
\newcommand{\CC}{\mathbb{C}}
\newcommand{\FF}{\mathbb{F}}
% Chemical constituents
\newcommand{\tF}{\text{F}}
\newcommand{\tX}{\text{X}}  % species "X"
\newcommand{\tCl}{\text{Cl}}
\newcommand{\tO}{\text{O}}  % oxygens
\newcommand{\tC}{\text{C}}  % carbon
\newcommand{\tN}{\text{N}}  % nitrogen
\newcommand{\tH}{\text{H}}  % hydrogen
\newcommand{\tS}{\text{S}}  % sulfur
\newcommand{\tM}{\text{M}}  % "third body"
\newcommand{\tR}{\text{R}}  % "hydrocarbon chain"
% Atmosphere common abbreviations
\newcommand{\TOA}{\text{TOA}}
\newcommand{\LW}{\text{LW}}
\newcommand{\SW}{\text{SW}}
\newcommand{\delt}{\partial_t}  % common partial derivatives
\newcommand{\deltt}{\partial_{tt}}
\newcommand{\delx}{\partial_x}
\newcommand{\dely}{\partial_y}
\newcommand{\delz}{\partial_z}
\newcommand{\delxx}{\partial_{xx}}
\newcommand{\delyy}{\partial_{yy}}
\newcommand{\delzz}{\partial_{zz}}
\newcommand{\delxy}{\partial_{xy}}
\newcommand{\delyx}{\partial_{yx}}
\newcommand{\delxz}{\partial_{xz}}
\newcommand{\delyz}{\partial_{yz}}
\newcommand{\deltx}{\partial_{tx}}
\newcommand{\delty}{\partial_{ty}}
\newcommand{\deltz}{\partial_{tz}}
\newcommand{\iter}[2]{#1^{(#2)}}  % iteration (for algorithm notation)
 in header (should store in home directory)
%
% \newcommand notes: CANNOT create a newcommand with same name as an environment;
% will get weird behavior; instead use some shortcut/abbrev

% Bib notes: 
% 0) use \bibliography INSIDE BODY to PRINT bibliography in natbib (I think need to also declare in preamble... or not? actually, no!), or use \printbibliography in biblatex
% 1) can also use NONE, and \cite{}, \bibliography{} commands work when you run BIBTEX COMMAND on the file... right??? and, so don't have to run bibtex on file otherwise?
% 2a) for smaller bibliographies, use \frame[shrink=50]{\printbibliography or \bibliography{file}} if you are using bibtex (is allowable, but remember, natbib [which uses bibtex as backend] does not work in beamer)
% 2b) smaller bibliographies in article, use {\footnotesize OR \small \bibliography} when using natbib 	
% 3) for some biblatex styles, need backend=biber instead of bibtex (biber is much newer/better than bibtex)... but these need different formatted .bib files
% 4) in Zotero, use Cmd+Shift+C to copy BibTeX citations onto clipboard -- seriously awesome!

%------------------------------------------------------------------------------
% MOST IMPORTANT/FOR PAPERS
%------------------------------------------------------------------------------
%% Must be changed OFTEN for different classes (e.g. publisher-supplied classes/templates
%% or Bibliography interpreters [AGU doesnt require natbib or biblatex], slides vs. article)
\usepackage[utf8]{inputenc} % disallows weird characters for unexpected behavior
\usepackage[T1]{fontenc} % ?
\usepackage{url} % package for formatting urls.
%\usepackage[backend=bibtex,style=alphabetic,citestyle=authoryear]{biblatex} % more "recent", and works for beamer (natbib  NOT work with slides)
%\usepackage{natbib} % allows citet, etc. and can use with custom .bst files; use for publications, and declare style with \bibliographystyle{<style>}
%\usepackage{biblatex} % basic, for beamer, etc.
\usepackage{lmodern} % this and ^^ declare font for whole article.
%\usepackage[margin=1in]{geometry} % geometry allows easy margin-formatting
\usepackage[labelfont=bf]{caption} % for better/more legible caption labels; very simple
\usepackage{subcaption} % for things like subfigure, etc.
%\usepackage{float} % special H location specifier is strong
\usepackage{graphicx} % embedded bitmap, vector graphics
\usepackage{epstopdf} % converts EPS to PDF files
\usepackage{url} % package for formatting urls.
%------------------------------------------------------------------------------
% PACKAGES
%------------------------------------------------------------------------------
%% Important Packages
% For math, etc.
\usepackage{amsmath}
\usepackage{mathtools} % fixes some amsmath quirks
\usepackage{amsthm}
\usepackage{amsopn}
\usepackage{amssymb}
\usepackage{amsfonts}
\usepackage{booktabs} % for lines between equations, use \midrule within; also
  % required for Pandas DataFrame.to_latex() method
%% Optional packages, for smaller/less relavant things
\usepackage{multirow} % for arrays; look this up
% \usepackage{enumerate} % enumitem options, and \begin{enumerate}[I.] e.g.
\usepackage{enumitem} % more complex enumeration (e.g. step 1, step 2, ...)
\usepackage{cancel} % special "cancel" operator
\usepackage{nth} % nth{1} is 1^st, etc.
\usepackage{pdflscape} % lets you switch particular pages to landscape with \begin{landscape} / \end{landscape}. also sets pdf's "rotate" attribute to rotate the landscape'd page{\tiny }
%% Some more specialized packages; need to look these up again
%\usepackage{blkarray}
%\usepackage{relsize}
%\usepackage{pgffor} % iteration of procedures
%\usepackage{setspace} % enables setstretch amont other things
%\usepackage{xhfill}% http://ctan.org/pkg/xhfill
%\usepackage{titling} % title options
%\usepackage[retainorgcmds]{IEEEtrantools}
%\usepackage{doc} % special logo commands? or... glossary? what the heck is this one?

%------------------------------------------------------------------------------
% REMAINING SHORTHANDS THAT MAY BE USEFUL
%------------------------------------------------------------------------------
%% Simple math operators (determines proper formatting AND spacing)
% \DeclareMathOperator{\Lagr}{\mathcal{L}} % cursive L
\DeclareMathOperator{\Tr}{trace}
\DeclareMathOperator{\id}{id} % identity function? not sure
\DeclareMathOperator{\eff}{eff} % effective? not sure
\DeclareMathOperator{\range}{range}
\DeclareMathOperator{\nulls}{null}  % nullspace
\DeclareMathOperator{\rank}{rank} 
\DeclareMathOperator{\sign}{sign} % sign function
\DeclareMathOperator{\Span}{span} % span
\DeclareMathOperator{\im}{Im}
% \DeclareMathOperator{\d}{d}
\DeclareMathOperator{\Char}{char}
\DeclareMathOperator{\argmax}{argmax}
\DeclareMathOperator{\argmin}{argmin}
\DeclareMathOperator{\sinc}{sinc}
%% Equation formatting
\newcommand{\numberthis}{\addtocounter{equation}{1}\tag{\theequation}} % add to end of line
%% Options for assignments header standard format. capitalized start new lines and "Solution" prefix
% \newcommand{\sectionline}{\begin{center}\noindent\rule{\paperwidth/2}{0.4pt}\end{center}} % better section line
\newcommand{\prob}[2]{\subsection*{Question #1}\textit{#2}\noindent}
\newcommand{\subprob}[2]{\subsubsection*{Part #1)}\textit{#2}\noindent}
\newcommand{\Prob}[2]{\subsection*{Question #1}\textit{#2}\\\\\textbf{Solution}: }
\newcommand{\Subprob}[2]{\subsubsection*{Part #1)}\textit{#2}\\\\\textbf{Solution}: }
%% Constants/units
\newcommand{\R}{8.314kg\,m^2\,s^{-2}\,mol^{-1}\,K^{-1}}
\newcommand{\kb}{1.38{\times}10^{-23}m^2\,kg\,s^{-2}\,K^{-1}}
\newcommand{\NA}{6.022{\times}10^{23}molec\,mol^{-1}}
\newcommand{\Rd}{287m^2\,s^{-2}\,K^{-1}}
\newcommand{\cpd}{1004J\,kg^{-1}\,K^{-1}}
\newcommand{\cvd}{717J\,kg^{-1}\,K^{-1}}
\newcommand{\cpw}{4187J\,kg^{-1}\,K^{-1}}
\newcommand{\cpi}{2108J\,kg^{-1}\,K^{-1}}
\newcommand{\atm}{1.01325 {\times} 10^5kg\,m^{-1}\,s^{-2}}
\newcommand{\speryr}{3.15576\E{7}s\,yr^{-1}}
\newcommand{\Pa}{kg\,m^{-1}\,s^{-2}}
\newcommand{\J}{kg\,m^2\,s^{-2}}
\newcommand{\rate}{cm^3\,molec^{-1}\,s^{-1}} % rate constant
\newcommand{\nd}{molec\,cm^{-3}} % number density
\newcommand{\pH}{\text{pH}} % pH, without italics
%% Figures
\newcommand{\fig}[3]{%
  \begin{figure}[H]
  \centering 
  \includegraphics[width=#1]{#2}
  \caption{#3}
  \end{figure}%
  } % for papers; also, float package conflicts with AGUTeX sometimes
\newcommand{\Fig}[3]{%
  \begin{figure}[H]
  \centering 
  \makebox[\textwidth][c]{\includegraphics[width=#1]{#2}}
  \caption{#3}
  \end{figure}%
  } % enforces centering always, even if runs into margins
\newcommand{\FIG}[3]{%
  \begin{minipage}{\linewidth}
  \centering 
  \makebox[\textwidth][c]{\includegraphics[width=#1]{#2}}
  \captionof{figure}{#3} % can't just do figure
  \end{minipage}
  } % enforces centering and disallows floating
\newcommand{\hFig}[3]{%
\begin{figure}
  \centering
  \begin{subfigure}{.5\textwidth}
    \centering
    \includegraphics[width=\linewidth]{#1}
  \end{subfigure}% necessary comment here, so figures are put side-by-side
  \begin{subfigure}{.5\textwidth}
    \centering
    \includegraphics[width=\linewidth]{#2}
  \end{subfigure}
  \caption{#3}
\end{figure} % multiple Fig-style figures
}
\newcommand{\hFIG}[3]{%
  \begin{minipage}{\linewidth}
  \centering
  \begin{subfigure}{.5\textwidth}
    \centering
    \includegraphics[width=\linewidth]{#1}
  \end{subfigure}% necessary comment here, so figures are put side-by-side
  \begin{subfigure}{.5\textwidth}
    \centering
    \includegraphics[width=\linewidth]{#2}
  \end{subfigure}
    \captionof{figure}{#3}
\end{minipage} % multiple FIG-style figures
}
\newcommand{\vFig}[3]{%
\begin{figure}
  \centering
  \begin{subfigure}{\textwidth}
    \centering
    \includegraphics[width=\linewidth]{#1}
  \end{subfigure}\\% put figures vertically
  \begin{subfigure}{\textwidth}
    \centering
    \includegraphics[width=\linewidth]{#2}
  \end{subfigure}
  \caption{#3}
\end{figure} % multiple Fig-style figures
}
\newcommand{\vFIG}[3]{%
  \begin{minipage}{\linewidth}
  \centering
  \begin{subfigure}{\textwidth}
    \centering
    \includegraphics[width=\linewidth]{#1}
  \end{subfigure}\\% put figures vertically
  \begin{subfigure}{\textwidth}
    \centering
    \includegraphics[width=\linewidth]{#2}
  \end{subfigure}
    \captionof{figure}{#3}
\end{minipage} % multiple FIG-style figures
}



%------------------------------------------------------------------------------
% EVERYTHING ELSE PROBABLY USELESS
%------------------------------------------------------------------------------
%------------------------------------------------------------------------------
% OPERATORS/SHORTHANDS
%------------------------------------------------------------------------------
%% Derivatives
\newcommand{\DD}[1][]{\,\text{D}#1}
\newcommand{\dd}[1][]{\,\text{d}#1} % for dx, dy, etc.; it is an "operator" now
\newcommand{\Deriv}[3][]{\frac{\text{D}^{#1}#3}{\text{D} {#2}^{#1}}} %capital D
\newcommand{\pderiv}[3][]{\frac{\partial^{#1}#3}{\partial {#2}^{#1}}}
\newcommand{\deriv}[3][]{\frac{\text{d}^{#1}#3}{\text{d} {#2}^{#1}}}
\newcommand{\px}{\frac{\partial}{\partial x}}
\newcommand{\py}{\frac{\partial}{\partial y}}
\newcommand{\pz}{\frac{\partial}{\partial z}}
\newcommand{\pt}{\frac{\partial}{\partial t}}
\newcommand{\dx}{\frac{\text{d}}{\text{d} x}}
\newcommand{\dy}{\frac{\text{d}}{\text{d} y}}
\newcommand{\dz}{\frac{\text{d}}{\text{d} z}}
\newcommand{\dt}{\frac{\text{d}}{\text{d} t}}
\newcommand{\Dx}{\frac{\text{D}}{\text{D} x}}
\newcommand{\Dy}{\frac{\text{D}}{\text{D} y}}
\newcommand{\Dz}{\frac{\text{D}}{\text{D} z}}
\newcommand{\Dt}{\frac{\text{D}}{\text{D} t}}
% Complex operators
\newcommand{\tnorm}[1]{\left\|#1\right\|_2} % 2-norm (Euclidian norm)
\newcommand{\onorm}[1]{\left\|#1\right\|_1} % 1-norm (sum of absolute values)
\newcommand{\inorm}[1]{\left\|#1\right\|_\infty} % infinity norm
\newcommand{\limit}[2][\infty]{\lim_{#2\to#1} } % limit, with default (symbol) goes to Inf
% Misc shorthands
\newcommand{\E}[1]{{\times}10^{#1}}
\newcommand{\Sum}{\sum_{i=1}^n}
\newcommand{\overbar}[1]{\overline{#1}}
\renewcommand{\tilde}[1]{\widetilde{#1}} % tilde is too narrow
\renewcommand{\inf}{\infty} % don't even know what \inf is; just redefine it
\newcommand{\ditto}[1][.4pt]{\text{\xrfill{#1}~\textquotedbl~\xrfill{#1}}}
\newcommand{\ubar}{\bar{u}} % mean x wind
\newcommand{\vbar}{\bar{v}} % mean y wind
\newcommand{\xbar}{\bar{x}}
\newcommand{\ybar}{\bar{y}}
\newcommand{\zbar}{\bar{z}}
\newcommand{\ihat}{\hat{i}}
\newcommand{\jhat}{\hat{j}}
\newcommand{\khat}{\hat{k}}
\newcommand{\uhat}{\hat{u}} % unit x wind
\newcommand{\vhat}{\hat{v}} % unit y wind
\newcommand{\xhat}{\hat{x}} % unit x
\newcommand{\yhat}{\hat{y}} % unit y
\newcommand{\that}{\hat{t}} % unit
\newcommand{\fhat}{\hat{f}} % unit Coriolis
\newcommand{\nhat}{\hat{n}} % normal unit vector
\newcommand{\F }{{}^\circ F} % degrees Fahrenheit
\newcommand{\C }{{}^\circ C} % degrees Celcius
\newcommand{\Tau}{\mathrm{T}} % greek tau
\newcommand{\given}{~|~}
\newcommand{\comma}{,~}
\newcommand{\fa}{~\forall~}
\newcommand{\ex}{~\exists~}

%------------------------------------------------------------------------------
% ENVIRONMENTS, STYLES, FORMATTING
%------------------------------------------------------------------------------
% Make arrays more spread out
\renewcommand{\arraystretch}{1.75}
% New title option, more compact
\newcommand{\mytitle}[1]{\setstretch{.65}\maketitle}
\renewcommand{\label}[1]{\label{#1}\tag{#1}}
% Environments
\def\[#1\]{\begin{align*}#1\end{align*}} % turns \[ \] from \begin/end equation, to align
\renewcommand{\cases}[1]{\left\{\begin{array}{ll}#1\end{array}\right.}
\newcommand{\case}[1]{\left\{\begin{array}{ll}#1\end{array}\right.}
\newcommand{\pfrac}[2]{\left(\frac{#1}{#2}\right)} % fraction with parentheses
\renewcommand{\frac}[2]{\dfrac{#1}{#2}} % will always want display-fraction
\newcommand{\mat}[1]{\begin{pmatrix}#1\end{pmatrix}} % quick parentheses matrix
\newcommand{\Mat}[1]{\begin{bmatrix}#1\end{bmatrix}} % quick bracket matrix
\newcommand{\bullets}[1]{\begin{itemize}#1\end{itemize}}
% \newcommand{\numbers}[1]{\begin{enumerate}[1)]#1\end{enumerate}}
\newcommand{\numbers}[1]{\begin{enumerate}#1\end{enumerate}}
\renewcommand{\table}[2]{\begin{tabular}[#1]#2\end{tabular}}
\newcommand{\abc}[1]{\begin{enumerate}[a)]#1\end{enumerate}}
\renewcommand{\roman}[1]{\begin{enumerate}[I.]#1\end{enumerate}}
\newcommand{\terms}[1]{\begin{description}#1\end{description}}
% Styling shortcuts (math-mode and regular-mode)
\newcommand{\un}[1]{\underline{\smash{#1}}} % underline without making line wider
\renewcommand{\it}[1]{\textit{#1}} % italics
\renewcommand{\bf}[1]{\textbf{#1}} % bold
\newcommand{\tit}[1]{\textit{#1}} % italics
\newcommand{\tbf}[1]{\textbf{#1}} % bold
\newcommand{\mb}[1]{\mathbf{#1}} % math bold
\newcommand{\mcal}[1]{\mathcal{#1}} % math calligraphy
\renewcommand{\t}[1]{\text{#1}} % for quick text in math-mode
\newcommand{\tx}[1]{\text{#1}} % for quick text in math-mode
\newcommand{\red}{\color{red}} % enclose text in red color

%------------------------------------------------------------------------------
% SPECIAL
%------------------------------------------------------------------------------
% Field labels
\newcommand{\II}{\mathbb{I}} % Imaginary
\newcommand{\RR}{\mathbb{R}} % for Real numbers
\newcommand{\ZZ}{\mathbb{Z}} % for Integers
\newcommand{\NN}{\mathbb{N}} % et. cetera
\newcommand{\QQ}{\mathbb{Q}}
\newcommand{\CC}{\mathbb{C}}
\newcommand{\FF}{\mathbb{F}}
% Chemical constituents
\newcommand{\tF}{\text{F}}
\newcommand{\tX}{\text{X}} % species "X"
\newcommand{\tCl}{\text{Cl}}
\newcommand{\tO}{\text{O}} % oxygens
\newcommand{\tC}{\text{C}} % carbon
\newcommand{\tN}{\text{N}} % nitrogen
\newcommand{\tH}{\text{H}} % hydrogen
\newcommand{\tS}{\text{S}} % sulfur
\newcommand{\tM}{\text{M}} % "third body"
\newcommand{\tR}{\text{R}} % "hydrocarbon chain"
% Atmosphere common abbreviations
\newcommand{\TOA}{\text{TOA}}
\newcommand{\LW}{\text{LW}}
\newcommand{\SW}{\text{SW}}
\newcommand{\delt}{\partial_t} % common partial derivatives
\newcommand{\deltt}{\partial_{tt}}
\newcommand{\delx}{\partial_x}
\newcommand{\dely}{\partial_y}
\newcommand{\delz}{\partial_z}
\newcommand{\delxx}{\partial_{xx}}
\newcommand{\delyy}{\partial_{yy}}
\newcommand{\delzz}{\partial_{zz}}
\newcommand{\delxy}{\partial_{xy}}
\newcommand{\delyx}{\partial_{yx}}
\newcommand{\delxz}{\partial_{xz}}
\newcommand{\delyz}{\partial_{yz}}
\newcommand{\deltx}{\partial_{tx}}
\newcommand{\delty}{\partial_{ty}}
\newcommand{\deltz}{\partial_{tz}}
\newcommand{\iter}[2]{#1^{(#2)}} % iteration (for algorithm notation)

